\section{Related Work}
%Two of the more common solutions for interfacing FPGA based hardware accelerators are AXI bus-based solutions~\cite{vipin2014zycap, sadri2013energy, xillybus2018} for FPGA SoC systems (like in Xilinx Zynq) and PCle-based solutions~\cite{xillybus2018, vipin2014dyract, gong2014efficient, de2016fflink, jacobsen2015riffa} for stand-alone systems connected to a host CPU. 

%Typical PCIe-based solutions for interfacing FPGA based hardware accelerators can be found from~\cite{xillybus2018, vipin2014dyract, gong2014efficient, de2016fflink, jacobsen2015riffa}. Most of them achieves high bandwidth close to theoretical maximum.  

\subsection{Time-Multiplexed Overlays}

QuickDough~\cite{liu2015quickdough} was proposed to address FPGA design productivity issues. 
It is comprised of an array of nearest neighbor interconnected processing elements (PEs), referred to as SCGRA overlay.  
Application specific SCGRA overlay were implemented on Zynq, achieving a speedup of up to $9\times$ compared to the same application qunning on the Zynq ARM processor. 
The 250 MHz FU consists of an ALUm multiport data memory ($256\times$32 bits) and customizable depth instruction ROM (Supporting 72-bit instructions) resulting in significant BRAM utilization. 
Due to the excessive usage of BRAMs, it has limited scalability as a maximum implementation of $5\times$5 array can fit on Zynq, despite of the frequency degradation caused by the tight placement and route. 


VectorBlox MXP~\cite{severance2013embedded} is a commercial soft vector processor overlay targetting Altera or Xilinx FPGAs via the Avalon or AXI interfaces. 
Data transfer is handled by a double-buffered vector scratchpad and a dedicated DMA engine communicating with the scalar host processor via the AXI HP port.
MXP can operate at a maximum frequency of 110 MHz on a Xilinx XC7Z020 device with 16 vector lanes. 
It demonstrates up to $1000\times$ speedup over MicroBlaze. 


\subsection{PCIe-based Interface Solutions} 
Northwest Logic~\cite{nwlogic2018} and Xillybus~\cite{xillybus2018} are two representative commercial solutions which support PCIe interfaces for different generations while providing portability across different FPGA devices. 
Northwest Logic is licensed closed source while Xillybus is provided free of charge for research and teaching purposes. 
Xillybus users have full access to customize the complete Xillybus project, including the hardware design, the device driver and the software applications, with good documentation provided.
The latest version of Xillybus (Revision XL) can achieve a maximum data rate of around 3.5 GB/s for Gen2x8 and Gen3x4 PCIe interfaces.

In addition to these commercial solutions, there are a number of academic PCIe-based solutions, such as DyRACT~\cite{vipin2014dyract} and EPEE~\cite{gong2014efficient}.
DyRACT mainly focuses on dynamic partial reconfiguration over the PCIe Gen2x4 interface, along with a configuration controller and clock management, while EPEE was developed as a general purpose PCIe communication library, targeting a wide range of FPGA devices. 

RIFFA~\cite{jacobsen2015riffa} is an open source reusable framework for the integration of FPGA accelerators with workstations supporting PCIe Gen 2 and Gen3 protocols. 
A scatter-gather DMA-based design bridges the vendor-specific PCIe Endpoint core and multiple communication channels for user defined IP cores. 
The latest release of RIFFA (version 2.2.2) achieves a unidirectional maximum bandwidth of 3.6 GB/s for the Gen2x8 configuration on the Xilinx VC707 platform and 3.5 GB/s for the Gen3x4 configuration on the Terasic DE5-Net.

Other recent solutions include ffLink~\cite{de2016fflink} and JetStream~\cite{vesper2016jetstream}. 
ffLink is an open source PCIe Gen3x8 interface which achieves a throughput of 7.06 GB/s on a Xilinx VC709 platform. 
ffLink is based on the AXI infrastructure with Xilinx CDMA engines, limiting its use beyond Xilinx devices. 
JetStream~\cite{vesper2016jetstream} provides both FPGA-to-Host and FPGA-to-FPGA communication, and supports PCIe Gen3x8  with a peak unidirectional bandwidth of 7.05 GB/s. 
However, poor documentation and tool compatibility issues limit its use.

%Table~\ref{theo_bw} shows the theoretical bandwidth of the AXI and PCIe memory interfaces. 
%Although the total theoretical bandwidth of an AXI interface is comparable to that of a PCIe interface, many of the AXI-based solutions are not able to make full use of all available high performance HP/ACP ports, while the PCIe-based counterparts can support up to 8 lanes and some of them achieve a bandwidth which is close to the theoretical maximum. 
Though TM overlays have shown great potential to impove the FPGA design productivity, few of them have been developed as full accelerator systems regardless of the existing work investivated above. 
The PCIe-based memory interface provides a perfect bridge between the host processor and the FPGA-based accelerators, due to its high bandwidth and low latency. 
Among the existing implementations, Xillybus, RIFFA, and DyRACT appear to be the most promising solutions due to their availability, ease-of-use and portability. 
In subsequent sections we develop high performance overlay-based accelerators designs using these solutions and then compare their capabilities.


\begin{comment}
\begin{table}[tb]
	%	\renewcommand{\arraystretch}{1.2}
	\caption{Theoretical bandwidth of typical memory interfaces.}
	\label{theo_bw}
	\centering
%	\small
	\resizebox{\columnwidth}{!}{
		\begin{threeparttable}
			\begin{tabular}{lrrrcr}
				\toprule
				Interface & Frequency & Upstream BW & Downstream BW & Ports/Lanes & Total BW\tnote{1} \\ \midrule
				AXI HP    &   150 MHz &   1200 MB/s &     1200 MB/s &      4      &         9600 MB/s \\
				AXI ACP   &   150 MHz &   1200 MB/s &     1200 MB/s &      1      &         2400 MB/s \\
				PCIe Gen2 &   250 MHz &    500 MB/s &      500 MB/s &      8      &         8000 MB/s \\
				PCIe Gen3 &   250 MHz &    984 MB/s &      984 MB/s &      8      &        15800 MB/s \\ \bottomrule
			\end{tabular} 
			\begin{tablenotes}
				\item[1] In a streaming interface, the throughput is limited by either the upstream or downstream BW, depending on the number of inputs or outputs, and so the maximum theoretical throughput would be half of the total bandwidth reported here.
			\end{tablenotes}
		\end{threeparttable}
	}
\end{table}
\end{comment}