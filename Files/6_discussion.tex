%\section{Discussion}
%We saw that AXI-Xillybus-V3 represents a very area efficient implementation compared with VectorBlox MXP-V16, at the expense of around half of the throughput. However,  AXI-Xillybus has only half the theoretical bandwidth of MXP, as Xillybus uses just 32-bits of an ACP port while MXP uses the full 64-bits of an HP port. If Xillybus were modified to use a 64-bit port (something that will be left for future work), and two parallel V3 overlays were implemented, both implementations would then have similar throughput. The AXI-Xillybus dual V3 implementation would require significantly fewer hardware resources (approximately 20\% of the LUTS and 8\% of the hard macros (BRAM and DSP) required by MXP on Zynq. Thus, the linear TM overlay represents a relatively efficient implementation when FPGA resources are limited, as would be the case when an accelerator was used with other major subsystems in a design.


%PCIe-Xillybus-V3 and RIFFA-V3 are proposed for more high-performance centric accelerator systems. The PCIe-Xillybus-V3 has a 1.6$\times$ better performance compared to the AXI-Xillybus-V3, at the expense of more than twice of resource consumption. However, the throughput of PCIe-Xillybus-V3 is slightly less than that of MXP-V16, as it achieves only 9\% of the theoretical maximum of the PCIe Gen2x8 interface. Among all the 4 implementations, RIFFA-V3 shows the best performance, with a throughput of 1300 MB/s on average. However, the half-duplex mode of RIFFA-V3 results in an over-utilization of BRAMs.
