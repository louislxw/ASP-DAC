\lstset { %
	language=C,
	backgroundcolor=\color{white}, % set backgroundcolor
	%basicstyle=\footnotesize,% basic font setting
	%basicstyle=\ttfamily\tiny,
	basicstyle=\ttfamily\small,
	keywordstyle=\color{blue}\ttfamily,
	stringstyle=\color{blue}\ttfamily,
	commentstyle=\color{green}\ttfamily,
	breaklines=true	
}
\lstset{framesep=-10pt, xleftmargin=-10pt}

\begin{table}[!h]
	\centering
	\caption{Example code.}
	\label{program}
	\begin{tabular}{l}
		%    \toprule
		%		\multicolumn{1}{c}{(a) C description}  &\multicolumn{1}{c}{(b) DFG description} \\ % Assembly with Loopback Optimization
		%    \midrule
		%    \hspace{-0.2in}
		%    \hspace{-0.2in}
		\begin{lstlisting}
fpga_reg_wr(0x30,0x0); //Tag of FU0 
fpga_reg_wr(0x34,0x3033D080); // Instruction 0
		
fpga_reg_wr(0x30,0x1); //Tag of FU1
fpga_reg_wr(0x34,0x8852000); // Instruction 1
		
fpga_reg_wr(0x38,5); // No. of input data
fpga_reg_wr(0x38,5); // (II-1)
		
dyract_send_data((unsigned char *)mydata, sendSize*sizeof(int)); //Send data
dyract_recv_data((unsigned char *) recvdata, recvSize*sizeof(4)); //Receive data
		\end{lstlisting}\\
		%    \bottomrule
	\end{tabular}
\end{table}
