\lstset{%
	% Basic design
	backgroundcolor=\color{lightgray},
	basicstyle={\scriptsize\ttfamily},   
	frame=l,
	% Line numbers
	xleftmargin={0.75cm},
	numbers=left,
	stepnumber=1,
	firstnumber=1,
	numberfirstline=true,
	% Code design
	identifierstyle=\color{black},
	keywordstyle=\color{blue}\bfseries,
	ndkeywordstyle=\color{greenCode}\bfseries,
	stringstyle=\color{ocherCode}\ttfamily,
	commentstyle=\color{darkgray}\ttfamily,
	% Code
	%	language={Python},
	language={C},
	tabsize=3,
	showtabs=false,
	showspaces=false,
	showstringspaces=false,
	extendedchars=true,
	breaklines=true
}
\begin{figure}
	\centering	
	\begin{lstlisting}[columns=fullflexible, language=C]
	#include <stdio.h>
	#include <stdlib.h>
	#include <errno.h>
	#include <fcntl.h>
	#include <unistd.h>
	#include <sys/types.h>
	#include <sys/stat.h>
	#include <sys/time.h>
	#include <pthread.h>
	
	#define VP void*
	#define NTH 2	
	int fdr32 = 0;
	int fdw32 = 0;
	int N = 0;
	int *array_input;
	int *array_hardware;
	struct timeval tv1, tv2;
	VP sample_1(VP arg) {
	write(fdw32, array_input, sizeof(int)*N);
	write(fdw32, NULL, 0);
	pthread_exit(NULL);
	}
	VP sample_2(VP arg) {
	read(fdr32, array_hardware, sizeof(int)*N);
	pthread_exit(NULL);
	}
	int main(int argc, char *argv[]) {
	fdr32 = open("/dev/xillybus_read_32", O_RDONLY);
	fdw32 = open("/dev/xillybus_write_32", O_WRONLY);
	N = atoi(argv[1]);
	if (fdr32 < 0 || fdw32 < 0) {
	perror("Failed to open devfiles");
	exit(1);
	}
	//allocate memory
	array_input = (int*)malloc(N*sizeof(int));
	array_hardware = (int*)malloc(N*sizeof(int));
	//initialize the inputs and outputs
	for(int i = 0; i < N; i++){
	array_input[i] = i;
	array_hardware[i] = 0;
	}
	// Creation of threads
	pthread_t tid[NTH];
	int value[NTH] = {1,2};
	pthread_create(&tid[0], NULL, &sample_1, &value[0]);
	pthread_create(&tid[1], NULL, &sample_2, &value[1]);
	// Synch of threads in order to exit normally
	gettimeofday(&tv1, NULL);
	for(int loop = 0; loop < NTH; loop++) {
	pthread_join(tid[loop], NULL);
	}
	gettimeofday(&tv2, NULL);
	for(i=0; i<N; i++){
	if(array_input[i] != array_hardware[i]) {
		printf("recv[%d]: %d , expected %d \n\r", i, array_hardware[i], array_input[i]);
		return;
		}
	}
	printf("round-trip bw: %f MB/s\n", N*4.0*2/1024/1024/((tv2.tv_sec-tv1.tv_sec)+(tv2.tv_usec-tv1.tv_usec)/1000000);
	return;
	}
	\end{lstlisting}
	\caption{Xillybus multi-threading program in C.}
	\label{xillybus_host_program_pthread}
\end{figure}